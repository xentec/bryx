\documentclass[12pt,a4paper]{article}
%\documentclass[12pt,a4paper,bibliography=totocnumbered,listof=totocnumbered]{scrartcl}
\input{includes}

\begin{document}

\input{commands}
% ----------------------------------------------------------------------------------------------------------
% Titelseite
% ----------------------------------------------------------------------------------------------------------
\MyTitlepage[pics/gamefield02]{G01}{
\texttt{andrej.utz@st.oth-regensburg.de}\\
\texttt{simon1.goldmann@st.oth-regensburg.de}}
{27.06.\the\year} % FIXME optional: Gruppenlogo als PNG, Pflichtfelder: Gruppe, Authoren durch "\\" getrennt und Abgabedatum eingeben

\setcounter{page}{1} 
% ----------------------------------------------------------------------------------------------------------
% Inhaltsverzeichnis
% ----------------------------------------------------------------------------------------------------------
\tableofcontents
\pagebreak


% ----------------------------------------------------------------------------------------------------------
% Inhalt
% ----------------------------------------------------------------------------------------------------------
% Abstände Überschrift
\titlespacing{\section}{0pt}{12pt plus 4pt minus 2pt}{-6pt plus 2pt minus 2pt}
\titlespacing{\subsection}{0pt}{12pt plus 4pt minus 2pt}{-6pt plus 2pt minus 2pt}
\titlespacing{\subsubsection}{0pt}{12pt plus 4pt minus 2pt}{-6pt plus 2pt minus 2pt}

% Kopfzeile
\renewcommand{\sectionmark}[1]{\markright{#1}}
\renewcommand{\subsectionmark}[1]{}
\renewcommand{\subsubsectionmark}[1]{}
\lhead{Kapitel \thesection}
\rhead{\rightmark}

\onehalfspacing
\renewcommand{\thesection}{\arabic{section}}
\renewcommand{\theHsection}{\arabic{section}}
\setcounter{section}{0}
\pagenumbering{arabic}
\setcounter{page}{1}

% ----------------------------------------------------------------------------------
% Kapitel: Einleitung
% ----------------------------------------------------------------------------------
\section{Einleitung}
%Leiten Sie in diesem Abschnitt in das Fach YIMB und das zu erstellende Projekt ein. Beschreiben Sie kurz die Fragestellung, die in diesem Wahlpflichtfach gelöst werden soll. Stellen Sie Ihr vorhandenes Vorwissen dar, das in dieser Veranstaltung für Sie von Nutzen sein könnte/ist/war, etc.
YIMB oder \glqq Implementierung von Brettspielen am Beispiel ReversiXT\grqq{} stellt ein Gruppensoftwareprojekt dar, das für diverse Aspekte eines 
Informatikstudiengangs ein Praktikum bietet. Ein Programm, das eine Modifikation von Reversi - ReversiXT - implementiert, soll auch eine künstliche 
Intelligenz beinhalten, die sich über das Netzwerk mit anderen K.I. messen soll. Algorithmen und Datenstrukturen sowie Netzwerkprogrammierung sind 
damit die am häufigsten beanspruchten Gebiete der Informatik in der Umsetzung des Projekts.\newline
Wir haben YIMB als Y-Fach gewählt, da es -wie bereits erwähnt- mehrere wichtige Bereiche der Informatik abdeckt. Zudem bietet es die Möglichkeit unser Wissen in, für die Technische Informatik relevanten Programmiersprachen zu vertiefen. Auch die betreute, vollständige Umsetzung eines Projekts sollte sich für die Zukunft, unter anderem im Hinblick auf die kommende Bachelor-Arbeit positiv auswirken. Zusätzlich erfordert dieses Projekt, dass man sich in \glqq LaTeX\grqq einarbeitet.


In Reversi geht es darum die Spielsteine des Gegenspielers zu übernehmen. Diese werden übernommen, wenn sie durch die Steine des Gegners eingekesselt sind. Im ursprünglichen Reversi spielen zwei Spieler gegeneinander, in der Erweiterung hingegen sind bis zu acht Spieler beteiligt. Auch das quadratische Spielfeld ist in der Erweiterung stark modifiziert in Form als auch in Funktion. Zum einen werden neue Sonderfelder und Sondersteine eingeführt, zum anderen werden an Spielfeldrändern Transitionen eingebaut.\newline Zu diesen Bonusfeldern gehören Choice-, Inversions-, Expansions- und Bonusfelder. Diese Sonderfelder werden zunächst wie normale Felder behandelt, das heißt es wird zuerst ganz normal umgefärbt, und dann treten die jeweiligen Sondereffekte auf. Bei Choice-Feldern darf man die Steine mit einem beliebigen Spieler tauschen. Inversionsfelder färben die Spielsteine aller Spieler in die Farbe des jeweiligen nachfolgenden Spielers. Expansionsfelder sind Felder die durch einen übernehmbaren Nicht-Spieler Spielstein besetzt sind. Bonusfelder geben je nach Wahl eine zusätzliche Bombe oder einen Überschreibstein.\newline Die Sondersteine setzen sich aus Überschreibsteinen und Bomben -welche in der Bombenphase Verwendung finden- zusammen. Überschreibsteine können direkt auf besetzte Felder gesetzt werden. Dabei müssen sie, sofern es sich nicht um einen Expansionsstein handelt, weitere besetzte Felder umfärben.Die Bombenphase findet statt, sobald kein Spieler mehr Steine oder Überschreib-Steine setzen kann. In dieser Phase werden Bombensteine auf besetzte oder unbesetzte Spielfelder, dabei werden alle Felder in einem vorgegebenen Radius gelöscht. Jeder Spieler hat am Anfang eine Vorgegebene Anzahl an Bomben und Überschreibsteinen.
\newpage
% ----------------------------------------------------------------------------------
% Kapitel: Allgemeine Informationen
% ----------------------------------------------------------------------------------
\section{Allgemeine Informationen}

\subsection{Projektname}
Um dem Projekt eine persönliche Note zu verleihen wurde das Programm \glqq \textbf{Bryx}\grqq{} getauft.

\subsection{Team und Kommunikation}
%Beschreiben Sie in diesem Abschnitt Ihr Team. Welche Person hat welche Aufgaben wahrgenommen, wie wurden Aufgaben aufgeteilt und wie wurde kommuniziert, etc.
Das Team und die Aufgabenverteilung sehen wie folgt aus:
\begin{itemize}
	\item Andrej Utz: Entwickeln, Programmieren, Testen, Dokumentieren und Präsentieren
\end{itemize}

Ursprünglich bestand das Team aus einer weiteren Person: Georg Fichtl. Weil die von ihm erhoffte Anrechnung des YIMB-Faches an das dritte Studienabschnitt durch formale Einschränkungen scheiterte, ließ er seine Motivation fallen und verließ das Team.

Am 10.05.2016 wurde Goldmann Simon ins Projekt aufgenommen, da sich seine vorherige Gruppe aufgelöste hatte.

Zur Kommunikation wird Discord (\url{https://discordapp.com}) benutzt, welches sich am Besten als eine Kombination eines IRC- und TeamSpeak-Clients
beschreiben lässt:

% TODO Besseres Bild
\begin{minipage}{\linewidth}
	\centering
	\includegraphics[width=0.6\linewidth]{pics/bryx-chat.png}
	\captionof{figure}[Discord-Chat]{Der \#bryx-Channel in Discord}
	\label{fig:discord}
\end{minipage}

Das persistente Chatprotokoll erlaubt Austausch und Festhaltung der Ideen und für längere Diskussionen kann man auch Sprachkommunikation nutzen.

% todo begründen
Programmversionen wurden mithilfe von Git ausgetauscht und verwaltet.

Die Bearbeitung der Aufgaben wurde wie folgt aufgeteilt:\newline
\begin{tabular}{|c|l|l|}
	\hline
	Übung & Aufgabe & bearbeitet von:\\
	\hline
	1.1 & Erstellen von einigen Spielfeldern & Utz Andrej\\
	1.2 & Implementieren einer Datenstruktur für Spielfelder & Utz Andrej\\
	1.3 & Einlesen von Spielfeldern & Utz Andrej\\
	\hline
	2.1 & Überprüfen von Zügen auf Gültigkeit & Utz Andrej\\
	2.2 & Ausformulieren von zwei Heuristiken & Goldmann Simon\\
	2.3 & Erweiterung der Überprüfung auf Gültigkeit & Utz Andrej\\
	\hline
	3.1 & Ausarbeiten einer Heuristik & Goldmann Simon, Utz Andrej\\
	3.2 & Implementieren der Ausgearbeiteten Heuristik & Goldmann Simon, Utz Andrej\\
	\hline
	4.1 & Implementierung der Netzwerkfunktionalität & Utz Andrej\\
	\hline
	5.1 & Implementierung des Paranoidsuchverfahrens & Utz Andrej\\
	\hline
	6.1 & Implementierung des $\alpha$-$\beta$-Pruning's & Utz Andrej\\
	6.2 & Erstellen von Statistiken zum $\alpha$-$\beta$-Pruning & To Be Done\\
	\hline
	7.1 & Einbauen der Client-Parameter & Utz Andrej\\
	7.2 & Implementierung des Iterative-Deepening-Verfahrens & Utz Andrej\\
	\hline
	8.1 & Implementieren des Aspiration-Windows-Verfahren & To Be Done\\
	\hline
	9.1 & Implementierung der Bombenphase & Utz Andrej\\
	\hline
	10.1 & Erstellen der Turnierspielfelder & Goldmann Simon\\
	\hline
	11.1 & Schreiben des Projektberichts & Goldmann Simon, Utz Andrej\\
	\hline
\end{tabular}
\newpage

\subsection{Technische Daten}
%Beschreiben Sie u.a.\ in welcher Programmiersprache und unter welchem Betriebssystem Sie entwickeln, welche IDEs Sie nutzen, welche zusätzlichen Tools bei Ihrer Projektentwicklung Einsatz gefunden haben, etc.
Bei jedem der Folgenden Einträge steht ein Vorteil dessen, der für die Wahl ausschlaggebend war.

\begin{itemize}
\item Programmiersprache: C++ - \url{http://cppreference.com} \\
Native Ausführung und deterministische Speicherverwaltung.
\item Compiler: clang - \url{http://clang.llvm.org} \\
Bessere Fehlermeldungen im Vergleich zu GCC.
\item Betriebssystem: \newline
Arch Linux - \url{https://www.archlinux.org} \\
Betriebssystem von Andrej Utz, da er bereits viel Erfahrung damit hat. \newline
Ubuntu 14.04 - \url{http://www.ubuntu.com} \\
Betriebssystem von Simon Goldmann, welches der Version auf dem Server entspricht.
\item Projektgenerator: CMake - \url{https://cmake.org} \\
Einfache und einheitliche Generierung der Projektdateien.
\item IDE: QtCreator - \url{http://wiki.qt.io/Qt_Creator} \\
Eine echte Open Source Alternative zu Visual Studio, die u.a. CMake versteht und dank Clang Static Analyzer semantische Fehler während des Schreibens anzeigt.
\item Dokumentationsgenerator: Doxygen - \url{http://www.stack.nl/~dimitri/doxygen} \\
De facto Standard zum Erstellen von Dokumentationen aus dem Quellcode.
\item Projektbericht-PDF:\newline TeXstudio - \url{http://www.texstudio.org/}\newline MiKTEX - \url{http://miktex.org/}\newline
Standard zum erstellen ordentlicher, größerer PDF-Dokumente.
\item Verwendete Programmbibliotheken:
\begin{itemize}
\item fmtlib (ehem. cppformat) - \url{http://fmtlib.net} \\
Typsichere Zeichenkettenformatierung und Konsolenausgabe.
\end{itemize}

\end{itemize}
\newpage

\subsection{Datenstruktur}
Als Grundlage für die Speicherung des Spielfeldes wurde der STL-Container \texttt{std::vector} aus der Standardbibliothek benutzt. Dieser ist nichts weiter als ein eindimensionaler, heapallokierter Array mit Verfolgung der Größe und Zugriffschecks. Der Zugriff auf Felder erfolgt über ihre X,Y-Koordinaten.
Ein Feld sieht wie folgt aus:

\begin{lstlisting}[caption=\texttt{Cell} Struktur, label=lst:cell-struct]
struct Cell
{
	struct Transition
	{
		Cell *target;
		Direction entry;
	};
	
	Map &map;
	const Vec2 pos;
	
	Cell::Type type;
	std::array<Transition, 8> transitions;
};
\end{lstlisting}

\texttt{Cell} speichert - neben dem eigentlichem Feldtyp - noch ihre Position als Vektor, eine Referenz zum Spielfeld und einen festen Array mit 8 Transitionen, welche den ihren Ausgangsrichtungen entsprechen.
Eine \texttt{Transition} beinhaltet einen Zeiger auf eine andere Zelle und die Ausrichtung, die ein Zug bekommt, nachdem er die Transition passiert hat.
Bei Nichtexistenz der \texttt{Transition} ist der Zeiger auf die \texttt{Cell} standardmäßig \texttt{NULL}.

Die Variante aus \texttt{std::vector} und \texttt{std::array} wurde genommen, damit die Felder beieinander angrenzen. Damit werden Cache-Misses der CPU beim sequentiellen Zugriff gemindert.\newline

\begin{minipage}{\linewidth}
	\centering
	\includegraphics[width=0.4\linewidth]{pics/BRYX-Cell.PNG}
	\captionof{figure}[Cell]{Abbildung einer Zelle(=Cell) und deren Umgebung}
	\label{fig:cell}
\end{minipage}

%Beschreiben Sie die Datenstruktur, die Sie zur Speicherung des Spielfeldes in Ihrem Client nutzen. Gehen Sie auf Besonderheiten ein und erklären Sie, wie diese funktionieren und was Sie sich davon erhoffen. Geben Sie falls möglich auch eine schematische Darstellung/ein Bild der Datenstruktur an.
\newpage
% ----------------------------------------------------------------------------------
% Kapitel: Spielfeldbewertung
% ----------------------------------------------------------------------------------
\section{Spielfeldbewertung}
\subsection{Spielfeldbewertung: 1. Vorschlag}
Bewertung durch entstehende Zugmöglichkeiten:\newline
Bei dieser Bewertungsvariante wird geprüft wie viele Spielzugmöglichkeiten für die Gegner, als auch für den setzenden Spieler entstehen. Auch die Anzahl der übernehmbaren Steine die durch diesen Spielzug entstehen sind für die Heuristik relevant. Ein Vorteil dieser Variante ist, dass man möglichst viele Steine übernehmen kann, ohne den Gegnern viele zu geben. Nachteilhaft ist die Implementierung dieser Bewertungsmethode, da sie sehr aufwendig ist.

\subsection{Spielfeldbewertung: 2. Vorschlag}
Bewertung nach Position: Hierbei wird geprüft, ob der gesetzte Stein durch normale Spielzüge übernehmbar ist oder nicht. Optional wäre es zu testen, wie viele normale Züge notwendig wären um einen übernehmbaren Stein zu drehen. Diese Bewertungsweise ist relativ leicht umsetzbar, aber nur mäßig gut um das Spielfeld zu bewerten.


\subsection{Final implementierte Spielfeldbewertung}
Nachdem das Spielfeld eingelesen wurde, erhält jedes Feld einen Wert, der davon abhängt, wie vielen Void-Felder angrenzen und ob diese nebeneinander liegen.

\begin{tabular}{|l|c|}
	\hline
	Beschreibung & Bewertung\\
	\hline
	Es liegen keine Void-Felder am zu bewertenden Feld an & 1\\
	\hline
	Keine Void-Felder liegen nebeneinander und & 2\\
	es grenzen eins bis vier Void-Felder am zu bewertenden Feld an &\\
	\hline
	Zwei Void-Felder liegen nebeneinander und & 3\\
	es grenzen zwei bis fünf Void-Felder am zu bewertenden Feld an &\\
	\hline
	Drei Void-Felder liegen nebeneinander und & 5\\
	es grenzen drei bis sechs Void-Felder am zu bewertenden Feld an &\\
	\hline
	Vier Void-Felder liegen nebeneinander und & 10\\
	es grenzen vier bis sechs Void-Felder am zu bewertenden Feld an &\\
	\hline
	Fünf Void-Felder liegen nebeneinander und & 7\\
	es grenzen fünf bis sechs Void-Felder am zu bewertenden Feld an &\\
	\hline
	Sechs Void-Felder liegen nebeneinander und & 4\\
	es grenzen sechs Void-Felder am zu bewertenden Feld an &\\
	\hline
	Sieben Void-Felder liegen nebeneinander und & 2\\
	es grenzen sieben Void-Felder am zu bewertenden Feld an &\\
	\hline
	Das zu bewertende Feld ist nur von Void-Feldern umgeben & 1\\
	\hline
\end{tabular}

Bei den angegebenen Bewertungen handelt es sich um Richtwerte, die sich in Abhängigkeit der Anzahl der angrenzenden Void-Felder ändert.\newline
Zudem wird zu bei jedem Nicht-Void-Feld geprüft, ob es sich um ein Bonus- oder Choice-Feld handelt. Sofern es sich um eines dieser Sonderfelder handelt wird ein vordefinierter Wert zum vorher ermittelten Feldwert hinzugefügt. Die Feldwerterhöhung verfällt sobald das Sonderfeld zum ersten Mal von einem Spieler besetzt wird.

\begin{tabular}{|l|c|}
	\hline
	Feldart & Bewertung\\
	\hline
	Choice-Feld & 5\\
	\hline
	Bonusfeld & 11\\
	\hline
\end{tabular}

Die Bewertung von Expansionssteinen und Inversionsfeldern wird durch das Paranoid-Verfahren bewältigt, da sie einen vordefinierten Einfluss auf das Spielfeld haben. Dieser Einfluss ist aus den, durch das Sonderfeld ausgelösten Umfärbungen ersichtlich.
\newpage
Sobald die Bewertung der Zellen durchgeführt ist, kann das Spiel -und somit die Bewertung der einzelnen Züge- beginnen. Die Zugbewertung erfolgt zunächst ähnlich wie die triviale Heuristik, die einfach nur Steine zählt. Es werden die Werte der eigenen Steine zusammenaddiert. Zusätzlich fließt die Anzahl der Überschreibsteine und Bomben mit ein. Somit wird es deutlicher, ob es sich lohnt Überschreibsteine zu einem bestimmten Zeitpunkt einzusetzen, da der \glqq Verlust\grqq eines Überschreibsteins eine konstante Wertabnahme bewirkt, die durch übernommene Steine ausgeglichen werden muss.
Bei Bonusfeldern wird der -zu Beginn bestimmte- Wert für Bomben mit dem konstanten Wert für Überschreibsteine verglichen und der höherwertige Bonusstein ausgewählt. Der Wert für Bomben wird wie folgt bestimmt:

\begin{equation*}
	\frac{100 \cdot (\text{Bombenstärke} \cdot 2 + 1)^2}{\text{Spielfeldbreite} \cdot \text{Spielfeldhöhe}}
\end{equation*}

%Beschreiben Sie abschließend, welche Heuristik final in Ihrem Client umgesetzt ist. Beschreiben Sie dazu auch Werte von Parametern (Kriterien und Gewichtungen), die Sie in den einzelnen Implementierungen nutzen. Welche statischen Vorberechnungen Sie machen, um z.B.\ das Spielfeld zu analysieren, etc.
\newpage
% ----------------------------------------------------------------------------------
% Kapitel: Statistiken
% ----------------------------------------------------------------------------------
\section{Statistiken}
%Integrieren Sie in diesen Abschnitt alle Ergebnisse von Projektaufgaben, die mit Erstellungen von Statistiken zu tun haben. Geben Sie dabei auch Diagramme an und interpretieren Sie die darin dargestellten Kurven. Beschreiben Sie zu jedem implementierten Verfahren, ob und welchen Nutzen es aus Ihrer Sicht gebracht hat.

\subsection{Vergleich ... und ...}

\newpage
% ----------------------------------------------------------------------------------
% Kapitel: Bombenphase
% ----------------------------------------------------------------------------------
\section{Bombenphase}
Die Bombenphase wird eingeleitet, sobald kein Spieler mehr Steine oder Überschreibsteine setzen kann. In dieser Phase werden Bomben auf Nicht-Void-Felder geworfen. Die Explosion breitet sich wellenförmig aus (ähnlich einem ins Wasser geworfenen Stein).Trifft eine Explosionswelle auf eine Transition, so breiten sich am anderen Ende der Transition die Wellen weiter aus. Diese Ausbreitung erweckt den Anschein, als ob an diesem Punkt im Spielfeld eine weitere -wenn auch kleinere- Bombe abgeworfen worden währe. Die Explosionswellen können keine Void-Felder überqueren, somit wird der Radius durch Void-Felder, an denen keine Transitionen liegen, eingeschränkt. Alles was innerhalb der Explosionswellen liegt wird in Void-Felder umgewandelt.

Bei unserem Projekt wird für jedes Nicht-Void-Feld berechnet, wie sich ein Bombenabwurf auswirken würde. Bei der Berechnung wird kontrolliert welche Felder von der Explosion betroffen werden, und ob es Felder von gegnerischen Spielern, leere- oder Void-Felder oder eigene Spielfelder sind. Eigene Felder wirken sich sehr negativ auf die Berechnung aus und werden mit einer -3 bepunktet, da es zu Vermeiden ist, eigene Felder zu zerstören und möglicher Weise seine Siegeschancen zu schmälern. Betroffene gegnerische Spielfelder fließen positiv in die Bewertung ein und sind mit +2 einberechnet. Je mehr gegner erwischt werden, desto besser. Lehre- oder Void-Felder sind vergeudetes Potential und sind somit mit -1 leicht negativ bepunktet.

Sobald die Bewertung durchgeführt ist, wird eine Bombe auf eine Möglichst gute Position, d.h. ein Feld mit einer möglichst hohen Punktezahl, geworfen. Danach sind die Gegner reihum an der Reihe ihre Bomben zu werfen, bis wir wieder an der Reihe sind. Dies wiederholt sich solange Spieler noch Bomben haben. (Logischer-Weise können nur Spieler Bomben werfen, die welche auf Lager haben, alle anderen werden einfach übersprungen.)
%Beschreiben Sie, wie Sie Bomben werfen (z.\,B.\ die eingesetzte Bewertungsheuristik und, ob Sie in die Tiefe rechnen und falls ja, wie tief Sie rechnen)
\newpage
% ----------------------------------------------------------------------------------
% Kapitel: Eigene Spielfelder
% ----------------------------------------------------------------------------------
\section{Wettbewerbs-Spielfelder}

%Beschreiben Sie in diesem Abschnitt die Spielfelder, die Sie für den Wettbewerb eingereicht haben/einreichen wollen. Fügen Sie in diesen Abschnitt auch die entsprechenden Bilder der Karten ein, geben Sie Zusatzinformationen wie Spieleranzahl, Bombenanzahl und -stärke, Anzahl Überschreibsteine etc.\ an.
%Beschreiben Sie außerdem, warum sie die jeweiligen Karten eingereicht haben: in welcher Hinsicht versprechen Sie sich von den eingereichten Karten Vorteile; in wie weit sind diese Karten auf Ihren Client und die darin implementierte Heuristik zugeschnitten, etc.
\newpage
% ----------------------------------------------------------------------------------
% Kapitel: Fazit
% ----------------------------------------------------------------------------------
\section{Fazit}

%Beschreiben Sie in diesem Abschnitt u.a.\ was Ihnen an diesem Fach gefallen hat und welche Verbesserungsvorschläge Sie für künftige Veranstaltungen haben. Was konnten Sie dazulernen, in welchen Bereichen haben Sie sich verbessert. Welche Problemsituationen gab es während der Projekterstellung, wie sind Sie diese angegangen und wie haben Sie diese gelöst. Was haben Sie evtl.\ vermisst.
\newpage
% ----------------------------------------------------------------------------------
% Kleine Einführung in LaTeX-Elemente
% ----------------------------------------------------------------------------------
%\section{\LaTeX-Elemente}
%Dieser Abschnitt soll nicht Bestandteil des Projektberichtes sein, sondern beinhaltet lediglich einige Informationen über \LaTeX-Distributionen, Editoren und \LaTeX-Elemente, die Ihnen beim Einstieg in das \LaTeX-Textsatzsystem helfen sollen.
%
%\subsection{\LaTeX-Distributionen nach Betriebssystemen}
%
%\subsubsection{\LaTeX-Distributionen}
%Folgende Haupt-\LaTeX-Distributionen stehen Ihnen zur Verfügung:
%\begin{itemize}
%  \item Windows:\quad \texttt{MiKTeX}\quad Webseite:\quad\url{http://www.miktex.org}
%  \item Linux/Unix:\quad \texttt{TeX Live}\quad Webseite:\quad\url{http://tug.org/texlive/}
%  \item Mac OS:\quad \texttt{MacTeX}\quad Webseite:\quad\url{http://www.tug.org/mactex/}
%\end{itemize}
%
%\subsubsection{\LaTeX-Editoren}
%Auf folgenden Webseiten können Sie einige hilfreiche \LaTeX-Editoren finden:
%\begin{itemize}
%  \item Windows/Linux/Mac OS: \url{http://www.xm1math.net/texmaker/}
%  \item Windiws: \url{http://www.texniccenter.org/}
%  \item Mac OS: \url{http://pages.uoregon.edu/koch/texshop/}
%\end{itemize}
%
%Falls bei den oben genannten Editoren kein passender vorhanden war, findet sich auf Wikipedia eine Zusammenstellung vieler weiterer \LaTeX-Editoren:\\[1em]
%\hspace*{3cm}\url{https://en.wikipedia.org/wiki/Comparison_of_TeX_editors}
%
%
%\subsection{Unterabschnitt}
%Zum Einfügen eines Bildes, siehe Abbildung \ref{fig:reversi01}, wird die \textit{minipage}-Umgebung genutzt, da die Bilder so gut positioniert werden können.
%
%\vspace{1em}
%\begin{minipage}{\linewidth}
%	\centering
%	\includegraphics[width=0.6\linewidth]{pics/gamefield01.png}
%	\captionof{figure}[Spielfeld 01]{Unbespieltes Spielfeld\footnotemark }
%	\label{fig:reversi01}
%\end{minipage}
%\footnotetext{Diesem Spielfeld wurden noch keine Spieler zugewiesen (daher die dunklen Spielsteine)}
%
%Nachdem das Spielt gestartet wurde und beiden Spielphasen durchlaufen wurden, siegt schließlich der Spieler mit der Farbe rot.
%
%\vspace{1em}
%\begin{minipage}{\linewidth}
%	\centering
%	\includegraphics[width=0.6\linewidth]{pics/gamefield02.png}
%	\captionof{figure}[Spielfeld 02]{Finales Spielfeld\footnotemark }
%	\label{fig:reversi2}
%\end{minipage}
%\footnotetext{Das Spielfeld nach der Zug- und Bombenphase. Spieler rot gewinnt eindeutig.}
%
%\subsection{Tabellen}
%In diesem Abschnitt wird eine Tabelle (siehe Tabelle \ref{tab:beispiel}) dargestellt.
%
%\vspace{1em}
%\begin{table}[!h]
%	\centering
%	\begin{tabular}{|l|l|l|}
%		\hline
%		\textbf{Name} & \textbf{Name} & \textbf{Name}\\
%		\hline
%		1 & 2 & 3\\
%		\hline
%		4 & 5 & 6\\
%		\hline
%		7 & 8 & 9\\
%		\hline
%	\end{tabular}
%	\caption{Beispieltabelle}
%	\label{tab:beispiel}
%\end{table}
%
%
%\subsection{Auflistung}
%Für Auflistungen wird die \textit{enumerate}- oder \textit{itemize}-Umgebung genutzt.
%
%\begin{itemize}
%	\item Nur
%	\item ein
%	\item Beispiel.
%\end{itemize}
%
%\subsection{Listings}
%Zuletzt ein Beispiel für ein Listing, in dem Quellcode eingebunden werden kann, siehe Listing \ref{lst:arduino}.
%
%\vspace{1em}
%\begin{lstlisting}[caption=Arduino Beispielprogramm, label=lst:arduino]
%int ledPin = 13;
%void setup() {
%    pinMode(ledPin, OUTPUT);
%}
%void loop() {
%    digitalWrite(ledPin, HIGH);
%    delay(500);
%    digitalWrite(ledPin, LOW);
%    delay(500);
%}
%\end{lstlisting}
%
%\subsection{Tipps}
%Die Quellen befinden sich in der Datei \textit{quellen.bib}. Eine Buch- und eine Online-Quelle sind beispielhaft eingefügt. [Vgl. \cite{buch}, \cite{online}]
%
%\pagebreak
%
%% ----------------------------------------------------------------------------------------------------------
%% Kapitel
%% ----------------------------------------------------------------------------------------------------------
%\section{Kapitel}
%Lorem ipsum dolor sit amet.
%
%\subsection{Unterkapitel}
%Lorem ipsum dolor sit amet, consetetur sadipscing elitr, sed diam nonumy eirmod tempor invidunt ut labore et dolore magna aliquyam erat, sed diam voluptua. At vero eos et accusam et justo duo dolores et ea rebum. Stet clita kasd gubergren, no sea takimata sanctus est Lorem ipsum dolor sit amet. Lorem ipsum dolor sit amet, consetetur sadipscing elitr, sed diam nonumy eirmod tempor invidunt ut labore et dolore magna aliquyam erat, sed diam voluptua. At vero eos et accusam et justo duo dolores et ea rebum. Stet clita kasd gubergren, no sea takimata sanctus est Lorem ipsum dolor sit amet.
%
%\subsection{Unterkapitel}
%Lorem ipsum dolor sit amet, consetetur sadipscing elitr, sed diam nonumy eirmod tempor invidunt ut labore et dolore magna aliquyam erat, sed diam voluptua. At vero eos et accusam et justo duo dolores et ea rebum. Stet clita kasd gubergren, no sea takimata sanctus est Lorem ipsum dolor sit amet. Lorem ipsum dolor sit amet, consetetur sadipscing elitr, sed diam nonumy eirmod tempor invidunt ut labore et dolore magna aliquyam erat, sed diam voluptua. At vero eos et accusam et justo duo dolores et ea rebum. Stet clita kasd gubergren, no sea takimata sanctus est Lorem ipsum dolor sit amet.
%\pagebreak
%
%% ----------------------------------------------------------------------------------------------------------
%% Literatur
%% ----------------------------------------------------------------------------------------------------------
%\renewcommand\refname{Quellenverzeichnis}
%\bibliographystyle{alpha}
%\bibliography{quellen}
%\pagebreak
%
%% ----------------------------------------------------------------------------------------------------------
%% Anhang
%% ----------------------------------------------------------------------------------------------------------
%\pagenumbering{Roman}
%\setcounter{page}{1}
%\lhead{Anhang \thesection}
%
%\begin{appendix}
%\section*{Anhang}
%\phantomsection
%\addcontentsline{toc}{section}{Anhang}
%\addtocontents{toc}{\vspace{-0.5em}}
%
%\section{GUI}
%Ein toller Anhang.
%
%\subsection*{Screenshot}
%\label{app:screenshot}
%Unterkategorie, die nicht im Inhaltsverzeichnis auftaucht.
%
%\end{appendix}
\end{document}
